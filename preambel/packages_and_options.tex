%%%
% Beschreibung:
% In dieser Datei werden zuerst die benoetigten Pakete eingebunden und
% danach diverse Optionen gesetzt. Achtung Reihenfolge ist entscheidend!
%
%%%


%%%
% Styleguide:
%
% Ein sehr kleiner Styleguide. Packages werden in Blöcken organisiert.
% Ein Block beginnt mit drei % in einer Zeile, dann % <Blocküberschrift>, dann
% eine Liste der möglichen Optionen und deren Einstellungen, Gründe und Kommentare
% eine % Zeile in der sonst nichts steht und dann wieder %%% in einer Zeile.
%
% Zwischen zwei Blöcken sind 2 Leerzeilen!
% Zu jedem Paket werden soviele Optionen wie möglich/nötig angegeben
%
%%%


%%%
% Codierung
% Wir sind im 21 Jahrhundert, utf-8 löst so viele Probleme.
%
% Mit UTF-8 funktionieren folgende Pakete nicht mehr. Bitte beachten!
%   * fancyvrb mit §
%   * easylist -> http://www.ctan.org/tex-archive/macros/latex/contrib/easylist/
\ifluatex
%no package loading required
\else
\usepackage[utf8]{inputenc}
\fi
%
%%%


%%%
% Deckblattstyle
%
\ifdeutsch
\PassOptionsToPackage{language=german}{uni-stuttgart-cs-cover}
\else
\PassOptionsToPackage{language=english}{uni-stuttgart-cs-cover}
\fi

\newcommand{\docauthor}{Jan Ruthardt\\
						Nico Poier\\
						Thomas Breunig}
\newcommand{\doctitle}{Evaluating Open-source Tool Stacks for Application Performance
	Diagnostics}

\usepackage[
title={\doctitle},
author={\docauthor},
type=Fachstudie,
institute=isterss,
number=12345,
course=se,
examiner={Dr.-Ing.\ Andr\'{e} van Hoorn (Prof.-Vertr.)},
supervisor={Thomas F. Düllmann M.Sc.,\\Teerat Pitakrat,\ M.Sc.},
startdate={August 14, 2017}, % English: July 5, 2013;    ISO: 2013-07-05
enddate={February 14, 2018}, % English: January 5, 2014; ISO: 2014-01-05
crk={I.7.2}
]{uni-stuttgart-cs-cover}
%
%%%


%%%
%Parallelbetrieb tex4ht und pdflatex
\makeatletter
\@ifpackageloaded{tex4ht}{\def\iftex4ht{\iftrue}}
                         {\def\iftex4ht{\iffalse}}
\makeatother
%%%


%%%
%Farbdefinitionen
\usepackage[hyperref,dvipsnames]{xcolor}
%


%%%
% Neue deutsche Rechtschreibung und Literatur statt "Literature", Nachfolger von ngerman.sty
\ifdeutsch
% letzte Sprache ist default, Einbindung von "american" ermöglicht \begin{otherlanguage}{amercian}...\end{otherlanguage} oder \foreignlanguage{american}{Text in American}
% see also http://tex.stackexchange.com/a/50638/9075
\usepackage[american,ngerman]{babel}
% Ein "abstract" ist eine "Kurzfassung", keine "Zusammenfassung"
\addto\captionsngerman{%
	\renewcommand\abstractname{Kurzfassung}%
}
\else
%
%
% if you are writing in english
% last language is the default language
\usepackage[ngerman,american]{babel}
\fi
%
%%%

%%%
% Anführungszeichen
% Zitate in \enquote{...} setzen, dann werden automatisch die richtigen Anführungszeichen verwendet.
\usepackage{csquotes}
%%%


%%%
% erweitertes Enumerate
\usepackage{paralist}
%
%%%


%%%
% fancyheadings (nicht nur) fuer koma
\usepackage[automark]{scrpage2}
%
%%%


%%%
%Mathematik
%
\usepackage[fleqn,leqno]{amsmath} % Viele Mathematik-Sachen: Doku: /usr/share/doc/texmf/latex/amsmath/amsldoc.dvi.gz
%fleqn (=Gleichungen linksbündig platzieren) funktioniert nicht direkt. Es muss noch ein Patch gemacht werden:
\addtolength\mathindent{1em}%work-around ams-math problem with align and 9 -> 10
\usepackage{mathtools} %fixes bugs in AMS math
%
%for theorems, replacement for amsthm
\usepackage[amsmath,hyperref]{ntheorem}
\theorempreskipamount 2ex plus1ex minus0.5ex
\theorempostskipamount 2ex plus1ex minus0.5ex
\theoremstyle{break}
\newtheorem{definition}{Definition}[section]
%
%%%


%%%
% Intelligentes Leerzeichen um hinter Abkürzungen die richtigen Abstände zu erhalten, auch leere.
% siehe commands.tex \gq{}
\usepackage{xspace}
%Macht \xspace und \enquote kompatibel
\makeatletter
\xspaceaddexceptions{\grqq \grq \csq@qclose@i \} }
\makeatother
%
%%%


%%%
% Anhang
\usepackage{appendix}
%[toc,page,title,header]
%
%%%


%%%
% Grafikeinbindungen
\usepackage{graphicx}%Parameter "pdftex" unnoetig
\graphicspath{{\getgraphicspath}}
\newcommand{\getgraphicspath}{graphics/}
%
%%%


%%%
% Enables inclusion of SVG graphics - 1:1 approach
% This is NOT the approach of http://www.ctan.org/tex-archive/info/svg-inkscape,
% which allows text in SVG to be typeset using LaTeX
% We just include the SVG as is
\usepackage{epstopdf}
\epstopdfDeclareGraphicsRule{.svg}{pdf}{.pdf}{%
  inkscape -z -D --file=#1 --export-pdf=\OutputFile
}
%
%%%


%%%
% Enables inclusion of SVG graphics - text-rendered-with-LaTeX-approach
% This is the approach of http://www.ctan.org/tex-archive/info/svg-inkscape,
\newcommand{\executeiffilenewer}[3]{%
\IfFileExists{#2}
{
%\message{file #2 exists}
\ifnum\pdfstrcmp{\pdffilemoddate{#1}}%
{\pdffilemoddate{#2}}>0%
{\immediate\write18{#3}}
\else
{%\message{file up to date #2}
}
\fi%
}{
%\message{file #2 doesn't exist}
%\message{argument: #3}
%\immediate\write18{echo "test" > xoutput.txt}
\immediate\write18{#3}
}
}
\newcommand{\includesvg}[1]{%
\executeiffilenewer{#1.svg}{#1.pdf}%
{
inkscape -z -D --file=\getgraphicspath#1.svg %
--export-pdf=\getgraphicspath#1.pdf --export-latex}%
\input{\getgraphicspath#1.pdf_tex}%
}


%%%
\usepackage{siunitx}
%%%

%%%
\usepackage{acro} %nice Acronyms
%\DeclareAcronym{TCT}{short = TCT , long = task completion time}
%%%


%%%
% Tabellenerweiterungen
\usepackage{array} %increases tex's buffer size and enables ``>'' in tablespecs
\usepackage{longtable}
\usepackage{dcolumn} %Aligning numbers by decimal points in table columns
\ifdeutsch
	\newcolumntype{d}[1]{D{.}{,}{#1}}
\else
	\newcolumntype{d}[1]{D{.}{.}{#1}}
\fi

%
%%%

%%%
% Eine Zelle, die sich über mehrere Zeilen erstreckt.
% Siehe Beispieltabelle in Kapitel 2
\usepackage{multirow}
%
%%%

%%%
%Fuer Tabellen mit Variablen Spaltenbreiten
%\usepackage{tabularx}
%\usepackage{tabulary}
%
%%%


%%%
% Links verhalten sich so, wie sie sollen
\usepackage{url}
%
%Use text font as url font, not the monospaced one
%see comments at http://tex.stackexchange.com/q/98463/9075
\urlstyle{same}
%
%Hint by http://tex.stackexchange.com/a/10419/9075
\makeatletter
\g@addto@macro{\UrlBreaks}{\UrlOrds}
\makeatother
%
%%%


%%%
% Index über Begriffe, Abkürzungen
%\usepackage{makeidx} makeidx ist out -> http://xindy.sf.net verwenden
%
%%%

%%%
%lustiger Hack fuer das Abkuerzungsverzeichnis
%nach latex durchlauf folgendes ausfuehren
%makeindex ausarbeitung.nlo -s nomencl.ist -o ausarbeitung.nls
%danach nochmal latex
%\usepackage{nomencl}
%    \let\abk\nomenclature %Deutsche Ueberschrift setzen
%          \renewcommand{\nomname}{List of Abbreviations}
%        %Punkte zw. Abkuerzung und Erklaerung
%          \setlength{\nomlabelwidth}{.2\hsize}
%          \renewcommand{\nomlabel}[1]{#1 \dotfill}
%        %Zeilenabstaende verkleinern
%          \setlength{\nomitemsep}{-\parsep}
%    \makenomenclature
%
%%%

%%%
% Logik für Tex
\usepackage{ifthen} %fuer if-then-else @ commands.tex
%
%%%


%%%
%
\usepackage{listings}
%
%%%


%%%
%Alternative zu Listings ist fancyvrb. Kann auch beides gleichzeitig benutzt werden.
\usepackage{fancyvrb}
%\fvset{fontsize=\small} %Groesse fuer den Fliesstext. Falls deaktiviert: \normalsize
%Funktioniert mit UTF-8 nicht mehr
%\DefineShortVerb{\§} %Somit kann im Text ganz einfach |verbatim| text gesetzt werden.
\RecustomVerbatimEnvironment{Verbatim}{Verbatim}{fontsize=\footnotesize}
\RecustomVerbatimCommand{\VerbatimInput}{VerbatimInput}{fontsize=\footnotesize}
%
%%%


%%%
% Bildunterschriften bei floats genauso formatieren wie bei Listings
% Anpassung wird unten bei den newfloat-Deklarationen vorgenommen
% https://www.ctan.org/pkg/caption2 is superseeded by this package.
\usepackage{caption}
%
%%%


%%%
% Ermoeglicht es, Abbildungen um 90 Grad zu drehen
% Alternatives Paket: rotating Allerdings wird hier nur das Bild gedreht, während bei lscape auch die PDF-Seite gedreht wird.
%Das Paket lscape dreht die Seite auch nicht
\usepackage{pdflscape}
%
%%%


%%%
% Fuer listings
% Wird für fancyvrb und für lstlistings verwendet
\usepackage{float}

%\usepackage{floatrow}
%% zustäzlich für den Paramter [H] = Floats WIRKLICH da wo sie deklariert wurden paltzieren - ganz ohne Kompromisse
% floatrow ist der Nachfolger von float
% Allerdings macht floatrow in manchen Konstellationen Probleme. Deshalb ist das Paket deaktiviert.
%
%%%



%%%
% Fuer Abbildungen innerhalb von Abbildungen
% Ersetzt das Paket subfigure
%
% Due to bug #24 in the caption package we need to update caption3.sty at the moment manualy to use subfig.
% Bug #24: http://sourceforge.net/p/latex-caption/tickets/24/
% corrected caption3.sty: http://sourceforge.net/p/latex-caption/code/HEAD/tree/branches/3.3/tex/caption3.sty
%
\usepackage[caption=false, lofdepth=1, lotdepth]{subfig}
%
%%%




%%%
% Fußnoten
%
%\usepackage{dblfnote}  %Zweispaltige Fußnoten
%
% Keine hochgestellten Ziffern in der Fußnote (KOMA-Script-spezifisch):
%\deffootnote[1.5em]{0pt}{1em}{\makebox[1.5em][l]{\bfseries\thefootnotemark}}
%
% Abstand zwischen Fußnoten vergrößern:
%\setlength{\footnotesep}{.85\baselineskip}
%
%
\renewcommand{\footnoterule}{}             % Keine Trennlinie zur Fußnote
\addtolength{\skip\footins}{\baselineskip} % Abstand Text <-> Fußnote
% Fußnoten immer ganz unten auf einer \raggedbottom-Seite
\usepackage{fnpos}
%
%%%


%%%
%
\raggedbottom     % Variable Seitenhöhen zulassen
%
%%%


%%%
% Falls die Seitenzahl bei einer Referenz auf eine Abbildung nur dann angegeben werden soll,
% falls sich die Abbildung nicht auf der selben Seite befindet...
\iftex4ht
%tex4ht does not work well with vref, therefore we emulate vref behavior
\newcommand{\vref}[1]{\ref{#1}}
\else
\ifdeutsch
\usepackage[ngerman]{varioref}
\else
\usepackage{varioref}
\fi
\fi
%%%

%%%
% Noch schoenere Tabellen als mit booktabs mit http://www.zvisionwelt.de/downloads.html
\usepackage{booktabs}
%
%\usepackage[section]{placeins}
%
%%%


%%%
%Fuer Graphiken. Allerdings funktioniert es nicht zusammen mit pdflatex
%\usepackage{gastex} % \tolarance kann dann nicht mehr umdefiniert werden
%
%%%


%%%
%
%\usepackage{multicol}
%\usepackage{setspace} % kollidiert mit diplomarbeit.sty
%
%http://www.tex.ac.uk/cgi-bin/texfaq2html?label=floats
%\usepackage{flafter} %floats IMMER nach ihrer Deklaration platzieren
%
%%%


%%%
%schoene TODOs
\usepackage{todonotes}
\let\xtodo\todo
\renewcommand{\todo}[1]{\xtodo[inline,color=black!5]{#1}}
\newcommand{\utodo}[1]{\xtodo[inline,color=green!5]{#1}}
\newcommand{\itodo}[1]{\xtodo[inline]{#1}}
%
%%%


%%%
%biblatex statt bibtex
\usepackage[
  backend       = biber, 	%biber dose not work with 64x versions alternative: bibtex8
														%minalphanames only works with biber backend														
  sortcites     = true,
  bibstyle      = alphabetic,
  citestyle     = alphabetic,
  firstinits    = true,
  useprefix     = false, %"von, van, etc." will be printed, too. See below.
  minnames      = 1,
  minalphanames = 3,
  maxalphanames = 4,
  maxbibnames   = 99,
  maxcitenames  = 3,
	natbib        = true,
	eprint        = true,
	url           = true,
  doi           = true,
  isbn          = true,
  backref       = true]{biblatex}
\bibliography{bibliography}
%\addbibresource[datatype=bibtex]{bibliography.bib}

%Do not put "vd" in the label, but put it at "\citeauthor"
%Source: http://tex.stackexchange.com/a/30277/9075
\makeatletter
\AtBeginDocument{\toggletrue{blx@useprefix}}
\AtBeginBibliography{\togglefalse{blx@useprefix}}

%Thin spaces between initials
%http://tex.stackexchange.com/a/11083/9075
\renewrobustcmd*{\bibinitdelim}{\,}

%Keep first and last name together in the bibliography
%http://tex.stackexchange.com/a/196192/9075
\renewcommand*\bibnamedelimc{\addnbspace}
\renewcommand*\bibnamedelimd{\addnbspace}

%Replace last "and" by comma in bibliography
%See http://tex.stackexchange.com/a/41532/9075
\AtBeginBibliography{%
  \renewcommand*{\finalnamedelim}{\addcomma\space}%
}

\DefineBibliographyStrings{ngerman}{
  backrefpage  = {zitiert auf S\adddot},
  backrefpages = {zitiert auf S\adddot},
  andothers    = {et\ \addabbrvspace al\adddot},
  %Tipp von http://www.mrunix.de/forums/showthread.php?64665-biblatex-Kann-%DCberschrift-vom-Inhaltsverzeichnis-nicht-%E4ndern&p=293656&viewfull=1#post293656
  bibliography = {Literaturverzeichnis}
}

%enable hyperlinked author names when using \citeauthor
%source: http://tex.stackexchange.com/a/75916/9075
\DeclareCiteCommand{\citeauthor}
  {\boolfalse{citetracker}%
   \boolfalse{pagetracker}%
   \usebibmacro{prenote}}
  {\ifciteindex
     {\indexnames{labelname}}
     {}%
   \printtext[bibhyperref]{\printnames{labelname}}}
  {\multicitedelim}
  {\usebibmacro{postnote}}

%natbib compatibility
%\newcommand{\citep}[1]{\cite{#1}}
%\newcommand{\citet}[1]{\citeauthor{#1} \cite{#1}}
%Beginning of sentence - analogous to cleveref - important for names such as "zur Muehlen"
%\newcommand{\Citep}[1]{\cite{#1}}
%\newcommand{\Citet}[1]{\Citeauthor{#1} \cite{#1}}
%%%


%%%
% Blindtext. Paket "blindtext" ist fortgeschritterner als "lipsum" und kann auch Mathematik im Text (http://texblog.org/2011/02/26/generating-dummy-textblindtext-with-latex-for-testing/)
% kantlipsum (https://www.ctan.org/tex-archive/macros/latex/contrib/kantlipsum) ist auch ganz nett, aber eben auch keine Mathematik
% Wird verwendet, um etwas Text zu erzeugen, um eine volle Seite wegen Layout zu sehen.
\usepackage[math]{blindtext}
%%%

%%%
% Neue Pakete bitte VOR hyperref einbinden. Insbesondere bei Verwendung des
% Pakets "index" wichtig, da sonst die Referenzierung nicht funktioniert.
% Für die Indizierung selbst ist unter http://xindy.sourceforge.net
% ein gutes Tool zu erhalten
%%%


%%%
%
% hier also neue packages einbinden
%
%%%


%%%
% ggf.in der Endversion komplett rausnehmen. dann auch \href in commands.tex aktivieren
% Alle Optionen nach \hypersetup verschoben, sonst crash
%
\usepackage[]{hyperref}%siehe auch: "Praktisches LaTeX" - www.itp.uni-hannover.de/~kreutzm
%
%% Da es mit KOMA 3 und xcolor zu Problemen mit den global Options kommt MÜSSEN die Optionen so gesetzt werden.
%

% Eigene Farbdefinitionen ohne die Namen des xcolor packages
\definecolor{darkblue}{rgb}{0,0,.5}
\definecolor{black}{rgb}{0,0,0}

\hypersetup{
    breaklinks=true,
    bookmarksnumbered=true,
    bookmarksopen=true,
    bookmarksopenlevel=1,
    breaklinks=true,
    colorlinks=true,
    pdfstartview=Fit,
    pdfpagelayout=SinglePage,
    %
    filecolor=darkblue,
    urlcolor=darkblue,
    linkcolor=black,
    citecolor=black
}
%
%%%


%%%
% cleveref für cref statt autoref, da cleveref auch bei Definitionen funktioniert
\ifdeutsch
\usepackage[ngerman,capitalise,nameinlink,noabbrev]{cleveref}
\else
\usepackage[capitalise,nameinlink,noabbrev]{cleveref}
\fi
%%%


%%%
% Zur Darstellung von Algorithmen
% Algorithm muss nach hyperref geladen werden
\usepackage[chapter]{algorithm}
\usepackage[]{algpseudocode}
%
%%%


%%%
% Schriften
\input{preambel/fonts}
%
%%%


%%%
% Links auf Gleitumgebungen springen nicht zur Beschriftung,
% Doc: http://mirror.ctan.org/tex-archive/macros/latex/contrib/oberdiek/hypcap.pdf
% sondern zum Anfang der Gleitumgebung
\usepackage[all]{hypcap}
%%%


%%%
%Bugfixes packages
%\usepackage{fixltx2e} %Fuer neueste LaTeX-Installationen nicht mehr benoetigt - bereinigte einige Ungereimtheiten, die auf Grund von Rueckwaertskompatibilitaet beibahlten wurden.
%\usepackage{mparhack} %Fixt die Position von marginpars (die in DAs selten bis gar nicht gebraucht werden}
%\usepackage{ellipsis} %Fixt die Abstaende vor \ldots. Wird wohl auch nicht benoetigt.
%
%%%


%%%
% Rand
\input{preambel/margins}
%
%%%


%%%
% Optionen
%
\captionsetup{
  format=hang,
  labelfont=bf,
  justification=justified,
  %single line captions should be centered, multiline captions justified
  singlelinecheck=true
}
%
%neue float Umgebung fuer Listings, die mittels fancyvrb gesetzt werden sollen
\floatstyle{ruled}
\newfloat{Listing}{tbp}{code}[chapter]
\crefname{Listing}{Listing}{Listings}
\newfloat{Algorithmus}{tbp}{alg}[chapter]
\ifdeutsch
\crefname{Algorithmus}{Algorithmus}{Algorithmus}
\else
\crefname{Algorithmus}{Algorithm}{Algorithms}
\fi
%
%amsmath
%\numberwithin{equation}{section}
%\renewcommand{\theequation}{\thesection.\Roman{equation}}
%
%pdftex
\pdfcompresslevel=9
%
%Tabellen (array.sty)
\setlength{\extrarowheight}{1pt}
%
%
%%%

%%%
% unterschiedliche Chapter-Styles
% u.a. Paket fncychap
\input{preambel/chapterheads}
%%%

%%%
%Minitoc-Einstellungen
%\dominitoc
%\renewcommand{\mtctitle}{Inhaltsverzeichnis dieses Kapitels}
%
% Disable single lines at the start of a paragraph (Schusterjungen)
\clubpenalty = 10000
%
% Disable single lines at the end of a paragraph (Hurenkinder)
\widowpenalty = 10000 \displaywidowpenalty = 10000
%
%http://groups.google.de/group/de.comp.text.tex/browse_thread/thread/f97da71d90442816/f5da290593fd647e?lnk=st&q=tolerance+emergencystretch&rnum=5&hl=de#f5da290593fd647e
%Mehr Infos unter http://www.tex.ac.uk/cgi-bin/texfaq2html?label=overfull
\tolerance=2000
\setlength{\emergencystretch}{3pt}   % kann man evtl. auf 20 erhoehen
\setlength{\hfuzz}{1pt}
%
%%%


%%%
% Fuer listings.sty
\lstset{language=XML,
        showstringspaces=false,
        extendedchars=true,
        basicstyle=\footnotesize\ttfamily,
        commentstyle=\slshape,
        stringstyle=\ttfamily, %Original: \rmfamily, damit werden die Strings im Quellcode hervorgehoben. Zusaetzlich evtl.: \scshape oder \rmfamily durch \ttfamily ersetzen. Dann sieht's aus, wie bei fancyvrb
        breaklines=true,
        breakatwhitespace=true,
        columns=flexible,
        aboveskip=0mm, %deaktivieren, falls man lstlistings direkt als floating object benutzt (\begin{lstlisting}[float,...])
        belowskip=0mm, %deaktivieren, falls man lstlistings direkt als floating object benutzt (\begin{lstlisting}[float,...])
        captionpos=b
}
\ifdeutsch
\renewcommand{\lstlistlistingname}{Verzeichnis der Listings}
\fi
%
%%%


%%%
%fuer algorithm.sty: - falls Deutsch und nicht Englisch. Falls Englisch als Sprache gewählt wurde, bitte die folgenden beiden Zeilen auskommentieren.
\floatname{algorithm}{Algorithmus}
\ifdeutsch
\renewcommand{\listalgorithmname}{Verzeichnis der Algorithmen}
\fi
%
%%%


%%%
% Das Euro Zeichen
% Fuer Palatino (mathpazo.sty): richtiges Euro-Zeichen
% Alternative: \usepackage{eurosym}
\newcommand{\EUR}{\ppleuro}
%
%%%


%%%
%
% Float-placements - http://dcwww.camd.dtu.dk/~schiotz/comp/LatexTips/LatexTips.html#figplacement
% and http://people.cs.uu.nl/piet/floats/node1.html
\renewcommand{\topfraction}{0.85}
\renewcommand{\bottomfraction}{0.95}
\renewcommand{\textfraction}{0.1}
\renewcommand{\floatpagefraction}{0.75}
%\setcounter{totalnumber}{5}
%
%%%

%%%
%
% Bei Gleichungen nur dann die Nummer zeigen, wenn die Gleichung auch referenziert wird
%
% Funktioniert mit MiKTeX Stand 2012-01-13 nicht. Deshalb ist dieser Schalter deaktiviert.
%
%\mathtoolsset{showonlyrefs}
%
%%%


%%%
%ensure that floats covering a whole page are placed at the top of the page
%see http://tex.stackexchange.com/a/28565/9075
\makeatletter
\setlength{\@fptop}{0pt}
\setlength{\@fpbot}{0pt plus 1fil}
\makeatother
%%%


%%%
%Optischer Randausgleich
\usepackage{microtype}
%%%

%%%
%Package geometry to enlarge on page
%
%Source: http://www.howtotex.com/tips-tricks/change-margins-of-a-single-page/
%
%Normally, this should not be used as the typearea package calculates the margins perfectly
\usepackage[
  pass %just load the package and do not destory the work of typearea
]{geometry}
%%%
