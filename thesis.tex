% !TeX spellcheck = en_US
% Dieses Dokument muss mit PDFLatex gesetzt werden
% Vorteil: Grafiken koennen als jpg, png, ... verwendet werden
%          und die Links im Dokument sind auch gleich richtig
%
%Ermöglicht \\ bei der Titelseite (z.B. bei supervisor)
%Siehe https://github.com/latextemplates/uni-stuttgart-cs-cover/issues/4
\RequirePackage{kvoptions-patch}

%Englisch:
\let\ifdeutsch\iffalse
\let\ifenglisch\iftrue

%Deutsch:
%\let\ifdeutsch\iftrue
%\let\ifenglisch\iffalse

%
\ifdeutsch
	\PassOptionsToClass{numbers=noenddot}{scrbook}
\fi

%Warns about outdated packages and missing caption delcarations
%See https://www.ctan.org/pkg/nag
\RequirePackage[l2tabu, orthodox]{nag}

%Neue deutsche Trennmuster
%Siehe http://www.ctan.org/pkg/dehyph-exptl und http://projekte.dante.de/Trennmuster/WebHome
%Nur für pdflatex, nicht für lualatex
\RequirePackage{ifluatex}
\ifluatex
%do not load anything
\else
	\ifdeutsch
		\RequirePackage[ngerman=ngerman-x-latest]{hyphsubst}
	\fi
\fi

\documentclass[
               fontsize=12pt, %Default: 11pt, bei Linux Libertine zu klein zum Lesen
% BEGINN: Optionen für typearea
               paper=a4,
               twoside, % fuer die Betrachtung am Schirm ungeschickt
               BCOR=3mm, % Hack für BCOR (1.92 o.ä.), da bei BCOR2mm die Fuellpunkte beim Inhaltsverzeichnis falsch sind. Hack aber nicht mehr nötig: microtype für Verzeichnisse ausschalten hilft.
               DIV=13,   % je höher der DIV-Wert, desto mehr geht auf eine Seite. Gute werde sind zwischen DIV=12 und DIV=15
               headinclude=true,
               footinclude=false,
% ENDE: Optionen für typearea
%               titlepage,
               bibliography=totoc,
%               idxtotoc,   %Index ins Inhaltsverzeichnis
%                liststotoc, %List of X ins Inhaltsverzeichnis, mit liststotocnumbered werden die Abbildungsverzeichnisse nummeriert
               headsepline,
               cleardoublepage=empty,
               parskip=half,
%               draft    % um zu sehen, wo noch nachgebessert werden muss - wichtig, da Bindungskorrektur mit drin
               final   % ACHTUNG! - in pagestyle.tex noch Seitenstil anpassen
               ]{scrbook}

\input{preambel/packages_and_options}

%Der untere Rand darf "flattern"
\raggedbottom

%%%
% Wie tief wird das Inhaltsverzeichnis aufgeschlüsselt
% 0 --\chapter
% 1 --\section % fuer kuerzeres Inhaltsverzeichnis verwenden - oder minitoc benutzen
% 2 --\subsection
% 3 --\subsubsection
% 4 --\paragraph
\setcounter{tocdepth}{1}
%
%%%

\makeindex

%Angaben in die PDF-Infos uebernehmen
\makeatletter
%\hypersetup{
 %           pdftitle={\doctitle}, %Titel der Arbeit
 %           pdfauthor={\docauthor}, %Author
 %           pdfkeywords={}, % CR-Klassifikation und ggf. weitere Stichworte
 %           pdfsubject={}
%}
\makeatother


%%% acro
% alle Acronyme die verwendet werden kommen hier her.

\DeclareAcronym{ER}{short = ER , long = error rate}
\DeclareAcronym{FR}{short = FR , long = Fehlerrate}
\DeclareAcronym{cli}{short = cli , long = Command-Line}
\DeclareAcronym{api}{short = api , long = Application}

%\usepackage[table,xcdraw]{xcolor}
%%%
\usepackage{hyperref}
%\usepackage[backend=bibtex,style=authoryear]{biblatex}

%\addbibresource{bibliography.bib}
\begin{document}

%tex4ht-Konvertierung verschönern
\iftex4ht
% tell tex4ht to create picures also for formulas starting with '$'
% WARNING: a tex4ht run now takes forever!
\Configure{$}{\PicMath}{\EndPicMath}{} 
%$ % <- syntax highlighting fix for emacs
\Css{body {text-align:justify;}}

%conversion of .pdf to .png
\Configure{graphics*}  
         {pdf}  
         {\Needs{"convert \csname Gin@base\endcsname.pdf  
                               \csname Gin@base\endcsname.png"}%  
          \Picture[pict]{\csname Gin@base\endcsname.png}%  
         }  
\fi

%Tipp von http://goemonx.blogspot.de/2012/01/pdflatex-ligaturen-und-copynpaste.html
%siehe auch http://tex.stackexchange.com/questions/4397/make-ligatures-in-linux-libertine-copyable-and-searchable
%
%ONLY WORKS ON MiKTeX
%On other systems, download glyphtounicode.tex from http://pdftex.sarovar.org/misc/
%
\input glyphtounicode.tex
\pdfgentounicode=1

\VerbatimFootnotes %verbatim text in Fußnoten erlauben. Geht normalerweise nicht.

\input{macros/commands}
\pagenumbering{arabic}
\Titelblatt

%Eigener Seitenstil fuer die Kurzfassung und das Inhaltsverzeichnis
\deftripstyle{preamble}{}{}{}{}{}{\pagemark}
%Doku zu deftripstyle: scrguide.pdf
\pagestyle{preamble}
\renewcommand*{\chapterpagestyle}{preamble}

\section*{Abstract}
Growing systems produce an exponential overhead of microservices. Due the a lack of control, even small problems can lead into a failure of the system. These problems especially occur in microservice environment like Kubernetes, since the number of nodes is significantly greater than in a monolith system. Our primary goal is, to find open-source programs that provide monitoring solutions which cover the whole range of problems, including the four types of monitoring: collecting, querying/saving of the metric data, visualize and sending notifications to the user. We compare tools of the four types and evaluate the results. The biggest problem is, that there are often no official solutions available for all tools. Most are from programmers who provide the tools themselves on GitHub. Our research showed that some tools are much easier and more convenient to install in comparison to others. In general we discovered that most solutions are working but not every tool is really convenient to use with Kubernetes. In our opinion, there are only a few tools which provide a satisfying experience when it comes to usability, productivity and performance/efficiency.

\cleardoublepage




\tableofcontents


% !TeX spellcheck = en_US

\chapter{Introduction}
Nowadays its very Common in IT to have Distributed Systems in Location all over the Globe. To be able to provide the best user Experience its important to Monitor these Networks by only few People sitting in one or more Location. Important here is the Availability and Reliability also as the Response Time of the System. 

To tackle these kinds of Tasks Application Performance Management Tool were build. They are available in a wide range of Costs and Qualities. They differ a lot in their Architecture and style of tackling Problems. This is why we decided to make a Comparison of some large Open-Source tool stacks available on the Market 


\section*{Thesis Structure}
In the first part the Paper describes the general aspects of Monitoring and the general Metrics. The part also discusses the characteristics of the environments and their special interfaces. After the introduction the tools will be introduced on there own. As a conclusion to this work a General overview in form of some table is given. And a conclusion to the question of the best monitoring tool is given. 
\begin{description}
\item[Technical Data:] In this chapter all technical aspects of the Test environments and Tools are discussed. More over it provides our separation of the different types of tools 
\item[Tools:] All tested Tool stacks are listed and group by companies that developed and supports them.If every single tool on the stack is usable in stand alone there is an subsection of them. On the end of this Chapter is a short list of tool we tested and were not able to deploy on a cluster.
\item[Evaluation:]The tested tools compared in tables after there responsibility 
\item[Conclusion:] Our experience with the tools and a Answer to the question which stack is the best.
\section*{Goals}
Goal of the Study is to print out the benefits and disadvantages of the popular Open-Source Tools and Stacks  for monitoring available on the Market. The work wants to Illustrate the features and technologies of the tools to make it easier for the Reader to get an overview over the different Software approaches. In particular the tools will be tested in their ability to interact with modern Cloud technologies like Docker and Kubernetes. Furthermore they will be compared by their Ability to integrate in existing environments and support of common tools and Interfaces. Moreover the Cross Compatibility of the stacks will be tested to get the best out of the tool pool.
\end{description}

%\input{...further chapters...}
% !TeX spellcheck = en_US

\chapter{General Stack}
\label{chap:ch2}
\section{Collector}

\section{Database}

\section{Visualization}

\section{Alerting}

%Hier wird der Hauptteil stehen. Falls mehrere Kapitel gewünscht, entweder mehrmals \texttt{\textbackslash{}chapter} benutzen oder pro Kapitel eine eigene Datei anlegen und \texttt{ausarbeitung.tex} anpassen.

%LaTeX-Hinweise stehen in \cref{chap:latextipps}.

%noch etwas Fülltext
\blinddocument

% !TeX spellcheck = en_US
 
\chapter{Tools} 
In this chapter we talk about all tools we decided to test in our study because they fitted best the requirements of \cref{minimumrequire}. The first part of this chapter handles the tools we managed to install. We used a template to ease the comparison of our experience with the specific tools. The second part contains a list with tools that are not capable of running in Kubernetes. A quick explanation of what went wrong in the installation process is given.


\section{Searchlight (Icinga)}
\label{searchlight}
As the \cref{Icinga} Icinga described, we were not able to find a pure installation of Icinga for a cluster so we tried to install Searchlight as a backup plan.
Searchlight is a tool from the AppsCode Inc. \cite{appscode}, which is a company located in San Leandro, California. 
Searchlight is as many other monitoring tools written in the programming language Go.
\\
When first tried to install Searchlight over the yaml File we received the error from the Kubernetes cluster. There it says the tool is not able to bind the Port 8443. We could not figure out which application is using the port but as we tried to install the application on a fresh cluster it turned out working fine.
\subsection{Appearance}
%\includegraphics[width=1\linewidth]{Bilder/Searchlight_Icinga}\\
Icinga/Searchlight comes in a  discreet blue/white coloring. On the dashboard which represents the homepage of the application an overview of the implemented alerts is shown. The alerts are colored with green/yellow/red for ok/warning/error, so it is easy to recognize appearing problems.
 Other functions like history and configurations are located at the sidebar. Information about the monitored host and services are located in an extra row, presented on the right.
\subsection{Performance}
In our VM deployment the application has allocated 160MB of RAM under maximum load. The application itself is running very smooth even on low power systems. As we deployed some alerts, we noticed, that it took about 30 seconds to validate the alert and get a first status from the system. Thirty seconds after deployment, the system shows no reaction. But after this time everything worked fine.
\subsection{Interoperability}
On the GitHub page \cite{searchlight} of the program the developer claims to be able to send notification over email, SMS and chat.
In the guide there is no explanation what is meant by the term chat but the tool is able to send notifications to Slack \cite{slack} and Hipchat \cite{hipchat}. Notification over email can be send over SMTP without any third party software. To send SMS, a service like Twilio \cite{twilio} is needed.
On the page there is no hint that the tool is capable to interact over an interface with other notification than the given. Furthermore the developer gives no interfaces in at all for communication with other applications.
\subsection{Conclusion}
Searchlight is a light weighted  implementation of the famous monitoring tool Icinga. It runs well and has all basic features. As we took a deeper look into the software we were able to uncover its weaknesses. There are no advanced possibilities to connect Searchlight with other monitoring tools. Also there is no graphical user interface for creating alerts, which makes it very difficult to set it up. As the tool is mainly developed and updated by only two people, the support of new versions from Icinga for Kubernetes is limited. The fact that the company Appcode Inc. has no direct correlation with Icinga is another counter-argument for the tool. In a nutshell the tool is good for small Kubernetes clusters that want to monitor only a few basic metrics and have low available resources. 

\section{Prometheus}
\label{Prometheus} % to refernce in the document
The tool Prometheus is an open source project and is hosted by the Cloud Native Computing Foundation.
It is a single tool that comes with the features of a whole monitoring stack (without alerting).\\ Prometheus own collector uses client libraries, supported for different languages including GO, Java, Scala, Python, Ruby and more third-party libraries for various other languages. The collector is able to pull data from the server or cluster and can be accessed via HTTP request.
The project was originally built at Soundcloud \cite{soundcloud} and  is nowadays used by big companies like Jodel \cite{jodel}, Docker and Core OS \cite{prometeus}.
Prometheus also has an alert manager, which is an extra tool to send message including the cluster state via email, HipChat, PagerDuty, Pushover, Slack, OpsGenie and VictorOps. These alerts have to be set up by a configuration file. 
Because the project is community driven, there is no official repository for Prometheus on Kubernetes, but with a little research we managed to find a repository \cite{prometheus_kube} that provided different deployments for cloud environments.

\subsection{Installation}
We managed to install Prometheus in a way that we were able to get metrics from the  Kubernetes Api-server and can expose the metrics over a RESTful interface. Over this interface it is possible to connect Prometheus to an application outside or inside the cluster like Grafana, which is recommended for Prometheus by the developers. We were not able to install the Prometheus own alert manager on the cluster. Also we were not able to get metrics from the cluster-nodes because our Prometheus version is not capable of requesting over HTTPS (Hypertext Transfer Protocol Secure), which has to be used for the nodes. 
\subsection{Appearance}%Only for Visual tools
Prometheus provides a web user interface to present the collected data. It is designed in a very light weighted black and a white combination with some blue accents and only shows the necessary informations. The interface has no function to save the set of graphs that is selected and no option to authenticate, so that the data can be keeped secret. 
\subsection{Performance}
The tool is very quickly booted and shows no performance problems on the interface. A look on the stats shows that Prometheus is not very CPU dependent, it oscillates up and down because the collection data peaks as shown in \cref{fig:Prometeus_Cpu}.Based on the amount of memory usage,the tool is a little bit more demanding and takes up to 2GB of RAM. The high amount of RAM is allocated over a period of about 24 hours. 
\begin{figure}
\centering
\includegraphics[width=1\linewidth]{Bilder/Performance/Prometeus_Cpu}
\caption{CPU usage of Promtheus}
\label{fig:Prometeus_Cpu}
\end{figure}
\subsection{Interoperability}
It is highly recommended that Prometheus is used with a graphing and an alerting tool, due to the fact that it is just a collector and time series database with a query engine. In our deployment Grafana is used for both tasks, because it is the recommended tool from the Prometheus web page \cite{prometheus}. Exploiting Prometheus to any other tool is also possible as Prometheus provides the metrics that are collected in plain text over a RESTful interface.
\subsection{Conclusion}
Prometheus is very good as a collector for large Kubernetes clusters with hundreds or thousands of nodes. What makes it so good is the with-box monitoring approach, which provides way more metrics than a black-box monitoring. These fact can be used to detect problems in the system earlier and in much more detail so that the downtime can be reduced. On the other side Prometheus is not good for presenting this data and to throw alerts in case of failure.

\section{Zabbix}
\label{Zabbix} % to refernce in the document
% \url{https://dl.acm.org/citation.cfm?id=1883485} Linux Journal
\cite{Hernantes2015}
The tool Zabbix is an open-source tool and is developed by Zabbix LLC (Limited Liability Company) since 2005 \cite{zabbix}. Zabbix provides an all in one solution which includes an collector, a database, a visualization tool and an alert manager.
\subsection{Installation}
There is a repository from monitoringartist \cite{zabbix_kube} that provides diverse yaml files, that map the Zabbix client and server on a cluster structure. The single modules are scalable by the user afterwards in the cluster. By default 3 copies of the database engine and web user interface are deployed. In the repository are two types of Zabbix deployment. The first we tested uses a load balancer the second one uses the cloud network balancing engine from Google.
\subsection{Appearance}%Only for Visual tools
Zabbix come white a white and blue web interface with some red accents. It implements a tab system with sub tabs to navigate through the interface. The naming of the tabs is not as self explaining as it could be.  
\subsection{Performance}
The general performance of Zabbix is very good. The three parts take up to 1GB of RAM form the System and about 0.5 \% of the processor load. The minimum requirements of Zabbix are specified with Pentium 2 of 350Mhz and 256MB of Ram \cite{Marik2014}.
\subsection{Interoperability}
Zabbix provides a different API to connect to other entities. The web interface  provides a PHP script that accepts queries and can be configured to use authentications . Grafana has an extra plug-in which is connectable to this script and shows a optimized dashboard with all relevant informations. The tool is also capable of sending alerts over email, SMS, and jabber. Zabbix also gives the ability to create custom scripts that  can be execute in case of an alert. For example a command line script can be triggered over SSH. 
\subsection{Conclusion}
Zabbix is one of the best optimized and extended open-source solutions available. The installation was by far the simplest of all tested tools. The configuration is complex and the naming of the tabs is not at every position as expected. By default Zabbix scales over all of the available nodes but can be scaled up manually. The best way to use Zabbix is to pare it with a Grafana instance which provides the information from Zabbix tighter and with a better overview.

\section{ELK Stack}
\label{elk} % to refernce in the document
The ELK stack is developed by elastic \cite{elasticsearch}. 
The stack is a logging tool which is developed in Java and divided up into three parts. First part is logstash and beats with their sub tools like Metricbeat or Heartbeat for collecting. Second part is Elasticsearch which queries the data and makes them persistent and last part is Kibana the visualization tool of the stack.
\subsection{Logstash/Beats}
Logstash is described as an 
"server-side data processing pipeline" \cite{elasticsearch}. These piplines are able to process nearly every data (github webhooks, http push/pull, irc, imap ,etc.).
While collecting the data Logstash already transforms it into data structures for better retrieval.
Beats is a "single-purpose data shipper" \cite{elasticsearch} and can be understood as a collector.
\subsection{Elasticsearch}
\label{Elasticsearch}
"Elasticsearch is a distributed, RESTful search and analytics engine capable of solving a growing number of use cases" \cite{elasticsearch}. It makes the data persistent in JSON (JavaScript Object Notation) with its own Query Domain Specific Language. These files are named shards and can be spread over multiple servers.
\subsection{Kibana}
Kibana is a browser based open-source visualization plugin. It is under the Apache license and is written in JavaScript. It features a web interface with a user login. This gives the owner the power to control who is using which features. User logins can be either added manually or via LDAP.
Kibana offers a lot of different graphs including classic line graphs, data tables, area graphs, pie charts and bar graphs.
\subsection{Installation}
ELK Stack is deployable at the Kubernetes cluster. There is a solution on GitHub provided by Kubernetes. Just the deploy the given yaml-files and the tools are running.but the necessary configuration files are missing.
\subsection{Appearance}%Only for Visual tools
Kibana is the only tool in the complete Stack with a user interface. It is accessible over the homepage of the application. The site is starting with a discover page, where all logs are listed in a chronological order. The discover page also allows to filter these logs by key words. Managing tools like dev tools, time, monitoring dashboard, etc can be found on the left hand sidebar. The dashboard tab is for creating an own dashboard with the own preferences.
\subsection{Performance}
The tool ELK stack is booting very fast. As seen in \cref{fig:ELK_CPU}, the CPU usage is very high at the start, but when all service have finished starting, the docker has a low CPU usage. In contrast the RAM consumption at the beginning is low, but when it starts collecting log data the amount of used RAM is growing very fast (\cref{fig:ELK_MEM}). The growth was so big, that our Kubernetes cluster broke after a view days because the server was out of memory.
\begin{figure}
	\centering
	\includegraphics[width=1\linewidth]{Bilder/Performance/ELK_CPU}
	\caption{}
	\label{fig:ELK_CPU}
\end{figure}
\begin{figure}
	\centering
	\includegraphics[width=1\linewidth]{Bilder/Performance/ELK_MEM}
	\caption{}
	\label{fig:ELK_MEM}
\end{figure}
\subsection{Interoperability}
Elasticsearch also offers a tool to include alerts. The tool can send any alerts which you can query with Elasticsearch. On the official website, elastics mentioned they can send the notifications on E-Mail, PagerDuty, Slack and HipChat and it also offers an interface with a webhook-output to integrate any third-party-software for messaging. With this interface it might be easy to include sending alerts with SMS like earlier mentioned the program Twilio(\ref{searchlight}).
\subsection{Conclusion}
ELK stack is easy to install but not easy to configure. There is no reference how to properly integrate the tools with each other in Kubernetes. Therefore it is impossible to get the data from Elasticsearch in Kibana, which makes the tool difficult for Kubernetes.
\section{Grafana}
\label{grafana} % to refernce in the document
Grafana is a tool to visualize data collected about a system. It features different graphs, including graphs that changes over time, single stats for live feeds, tables to show multiple data from different sources and heat maps to show the distribution of values over a timespan.
Grafana also comes with an inbuilt alerting system which allows the user to be notified under specific conditions via E-Mail, Slack, PagerDuty and Kafka. The alerting also has a webhook, allowing the user to send his alerts to a custom endpoint via HTTP.
The graphs and alerts are setup in the web interface of Grafana which comes with its own user management. It differentiates between users, administrators and super-administrators. Administrators can be assigned to an organization, allowing different systems to be monitored by one Grafana instance, while still giving the owners of the single system the option to manage their own alerts and graphs.
Grafana also features a LDAP authentication to automatically import users and permissions from a user directory.
\subsection{Installation}
The installation of Grafana is very simple, because it only runs on a single Docker container. Grafana is not a collector. Because of this reason, there are many tools which offer an own distribution of Grafana in their repository that is optimized for their data sources.
\subsection{Appearance}%Only for Visual tools
Grafana mainly uses the colors red and orange and comes with a dashboard. Depending on the data source that is selected this changes to a set of graphs which show the information of the data source. Per drag and drop the graphs and text boxes can be freely moved around. 
\subsection{Performance}
The performance of Grafana is very good. The tool consumes nearly no CPU and only a few megabytes of RAM as shown in \cref{fig:Grafan_RAM}. The Diagram also shows that the amount of the used resources is constant over time
\begin{figure}
\centering
\includegraphics[width=1\textwidth]{Bilder/Performance/Grafan_RAM}
\caption{Grafana resource use}
\label{fig:Grafan_RAM}
\end{figure}
\subsection{Interoperability}
Grafana can interact with nearly every interface to sent alerts because of its web hooks which can send and request to an restful endpoint over a JSON document. Whats special about the alerting engine of Grafana is that pictures can be send within the alerts. The pictures are taken from Grafana to present the alert in a visual way and can be send with the notification or can be uploaded to services like Amazon S3.

\subsection{Conclusion}
Grafana is by far the best and biggest visualization tool on the open-source market. Due to the capability to write custom plug ins nearly every tool stack has an integration which is easy to install and uses the features of Grafana in the intention of the developer.

\section{TICK}
\label{tick}
TICK is a monitoring Stack provided by the InfluxData company \cite{tick} and is under MIT License which is a permissive software license.
\subsection{Telegraf}
Telegraf is the data collector of the TICK stack by InfluxData and is written entirely in GO. It is a lightweight data collector as it needs specific plugins for metrics to be collected. These can be chosen by the user as needed for the system. It has official input plug ins for many servers including Apache, Cassandra, Kubernetes, MySQL.
 This allows the user to send the metrics to almost any endpoint. As with the input plug ins, there is also a broad variety of output plugins including for Influxdatas own InfluxDB, Elasticsearch(\ref{Elasticsearch}) from the ELK Stack and Prometheus(\ref{Prometheus}). It also supports more primitive outputs like writing to file or TCP/UDP Socket.
\subsection{InfluxDB}
InfluxDB is the time-series database from the TICK stack by InfluxData. It is written in GO and available as a single binary with no external dependencies. This makes the installation, either as a standalone database or as a Kubernetes pod, simple.
By default it supports their own collector Telegraf, but other collectors are supported via plug ins. The data can be written with their  HTTP/S API using its own SQL-like query language, InfluxQL, which also gives the option of creating your own interfaces.
To minimize storage consumption data will be down sampled over time. The recent high precision data is stored in the database only for a limited time and then summarized with other data to keep hard-drive usage reasonable. 
\subsection{Chronograf}
Chronograf is the Visualization Tool of the TICK Stack by InfluxData. It comes with a web interface to display all the data collected by Telegraf. Data can be visualized in different graph-types including single and stack line graphs, Step-Plot, Single stat and bar graphs. It comes with a lot of pre-configured dashboards for all the applications which are also officially supported by Telegraf, though it is also possible to create your own dashboards or modify the existing ones. 
Chronograf also features a UI for InfluxData own alerting tool Kapacitor, which allows the management of both tools in one interface.
\subsection{Kapacitor}
Kapacitor is the alerting tool of the TICK Stack by InfluxData. Various alerts can already be defined when using Chronograf with Kapacitor including static thresholds, percentage values and deadman switches. It also supports more complex alert conditions via TICKscripts which have to be defined in Kapacitor directly.
The alerts itself can then be sent to various endpoints like HipChat, Pagerduty, Pushover, Slack, Telegram and many more. It also supports more simplistic endpoints like file outputs, TCP sockets or HTTP Post.
Kapacitor does not feature a web interface as it is meant to be used with Chronograf which already has a UI component for Kapacitor. It can be used without this interface via console inputs.
\subsection{Installation}
The installation of the TICK stack in special Telegraf and Influxdb is very simple. A possible portation of the tool is delivered by the developer of Kubernetes in there GitHub repository Heapster \cite{heapster}. Heapster is used by the stack to collect the monitoring information of the pods in one place.
\subsection{Appearance}
Only Chronograf provides a web interface for the user. The interface is in blue white combination and contains a lot of black. The menu is located on an left sidebar but there are only symbols. On hover the text appears and shows further informations for navigation. It offers a graphical way to inject alerts into Kibana.

\subsection{Performance}
In our cluster the TICK stack performed very well which matches the InfluxData guidelines that recommend a dual core CPU and 2 to 4 GB of RAM (see alos \cref{fig:TICK_recommendet}) as minimum and for system with under five queries per second. Due to the Developer TICK can handle up to over 100 queries but then need 4 to 8 times more resources \cite{influx_require}.
\begin{figure}
\centering
\includegraphics[width=0.35\textwidth]{Bilder/Performance/TICK_recommendet}
\caption{Grafana Ressource use}
\label{fig:TICK_recommendet}
\end{figure}

\subsection{Interoperability}
TICK and in special influxDB can be accessed by many tools. Because TICK stack is very common many tools offer plug-in integrations. In special we tested Grafana because it is recommended by the developer and integrates out of the box. Influx exposes to it not only the physical specs of all nodes, but also the stats collected by Kubernetes to every deployed pod. 
\subsection{Conclusion}
The TICK stack is one of the best optimized monitoring solutions for Kubernetes. What makes it special is that information about every running pod instance is collected and can be accessed.



\section{Failed Tools}%Ebenfalls nur ein arbeitstitel
In the process of developing and evaluating the APM we discovered a bunch of tools that we were not able to install even that they claimed to be optimized to work on Kubernetes .
In this paragraph all the tools we wanted to include in our report but did not work, are listed with a quick description of the failure.
\subsection{Graphite}
Graphite is mainly for storing and graphing data and metrics, but brings also tools that are able to collect these metrics from the system. By the developer himself there is no Kubernetes installation provided but there are diverse approaches by third party members to make it runnable on a Cluster. We have tested the Repository from nanit \cite{graphite} to get Graphite running with StatsD \cite{statsd} as a metric collection tool. The Repository does not provide a yaml file by itself to install all the tools. The instruction leads the user to export some variables needed for the installation. After that a deploy command is provided that pulls the docker repository and then installs it with kubectl on the Cluster. As we tried to execute this command an error was thrown. The node replicas were empty, so no further commands were executable. As we were not able to install the tool on multiple Kubernetes clusters, we installed the tool as on the website advertised on a Ubuntu system directly. With this installation of Graphite a time-series data, a monitoring and alerting tool is included. On the test system the monitoring system gave values that differ from the Linux intern monitoring in values like CPU usage or RAM. As a solution to all this difficulties we decided not to perform further test on this tool.

\subsection{Icinga}
\label{Icinga}
Icinga is as ELK-Stack not a monitoring tool. Its made for logging defined requests. The administrator can decide which system to monitor and in what interval the data are pulled.
Icinga it self only provides three system states instead of exact values. The states are ok,warning and error. The user decides at which point they are triggered.
As we tried to install Icinga we found out that there was no direct support from Incinga for Kubernetes. As our study only describes the actual state without trying to add something we decided not to try an compile Icinga into Kuberntes on our own. Later we found out there is a third party software called searchlight \ref{searchlight} which is provided by Appscode on GitHub.
 
% !TeX spellcheck = en_US
\chapter{Evaluation} %Das ist nur ein Arbeitstitel
% \useunder{\uline}{\ul}{}
\section{Supplier details of the tools}
The tables show the supplier details by the type of the tools. These were selected from the website of the tool and compared with selected critical points. 
\subsection{Collectors}
The collector table \ref{tab:Collector} shows all collector-tools deployed and tested in this document. 
\subsection{Databases}
The database table \ref{tab:database} shows all database-tools deployed and tested in this document.
\subsection{Visualization}
The database table \ref{tab:visualization} shows all visualization-tools deployed and tested in this document.
\subsection{Alerting}
The database table \ref{tab:alerting} shows all alerting-tools deployed and tested in this document.
\begin{table}
\centering
\label{tab:Collector}
\begin{tabular}{p{3cm}cccc}
\hline
Tool & RESTful                     & Multiple Input Formats      & Plugins for own data        & Push / Pull                 \\
\hline
Telegraf                     & Yes & Yes                         & Yes                         & Pull                         \\
Beats                        & Yes                         & Yes & Yes                         & Push                         \\
Logstash                     & Yes                         & No                          & Yes & Both                         \\
Icinga2 (Plugins)            & No                          & No                          & Yes                         & Push \\
Prometheus (Client)          & No                          & No                          & Yes                         & Push \\
Zabbix (Agent)               & No                          & No                          & No  & Both \\
\hline                        
\end{tabular}
\caption{Collectors}
\end{table}


% Please add the following required packages to your document preamble:
% \usepackage[table,xcdraw]{xcolor}
% If you use beamer only pass "xcolor=table" option, i.e. \documentclass[xcolor=table]{beamer}
\begin{table}
\centering

\label{tab:database}
\begin{tabular}{p{3cm}p{3cm}p{3cm}c}
	\hline
Tool & Database-Authentication     & Timestamp-Database          & Multiple Data Points        \\
\hline
InfluxDB                     & Yes & Yes                         & Yes                         \\
Elasticsearch                & Only https and credentials  & Yes & Yes                         \\
Icinga2 (Server)             & No                          & No                          & No                          \\
Prometheus (Master)          & No                          & Yes & Yes                         \\
Zabbix (Master)              & Yes                         & Yes                         & Yes\\
\hline
\end{tabular}
\caption{Databases}
\end{table}

% Please add the following required packages to your document preamble:
% \usepackage[table,xcdraw]{xcolor}
% If you use beamer only pass "xcolor=table" option, i.e. \documentclass[xcolor=table]{beamer}
\begin{table}
\centering

\label{tab:visualization}
\begin{tabular}{cccc}
	\hline
Tool & LDAP-Authetication         & Scalable Graphs             & Overlapping Graphs          \\
\hline
Chronograf                   & No & Yes                         & Yes                         \\
Kibana                       & Yes                        & Yes & Yes                         \\
Grafana                      & Yes                        & Yes & Yes                         \\
Zabbix (Master)              & Yes                        & Yes                         & Yes\\
\hline
\end{tabular}
\caption{Visualization Tools}
\end{table}

\begin{table}
	\centering
	
	\label{tab:alerting}
	\begin{tabular}{lcccc}
		\hline
Tool & HTTP requests & Grouping of alerts         & E-Mail                      & Messangers \\
\hline
Kapacitor                    & Yes           & No & Yes                         & Yes \\
Elastalert & No & No & No & No \\
Grafana                      & Yes           & Yes                        & Yes & Yes \\
Zabbix (Master)              & Yes           & Yes                        & Yes                         & Yes \\
		\hline
	\end{tabular}
	\caption{Alerting Tools}
\end{table}

% Please add the following required packages to your document preamble:
% \usepackage[table,xcdraw]{xcolor}
% If you use beamer only pass "xcolor=table" option, i.e. \documentclass[xcolor=table]{beamer}
\begin{table}
	\centering
	
	\label{tab:alertingcont}
	\begin{tabular}{lcc}
		\hline
		Tool & Custom alerts & Action on alert \\
		\hline
		Kapacitor                    & Yes           & No              \\
		Elastalert                   & Yes           & No              \\
		Grafana                      & Yes           & Yes             \\
		Zabbix (Master)              & Yes           & Yes           \\
		\hline 
	\end{tabular}
	\caption{Alerting Tools cont.}
\end{table}
% !TeX spellcheck = en_US

\chapter{Conclusion}\label{chap:conclusion}
\section{General Trends}
Most of the systems we evaluated or considered evaluating were metric monitoring stacks. In general most tools detect not the cause of the failure. Instead the effect is recognized and can be treated manual. With some tools we saw a trend towards more automation. These tools had functions like automated script triggering over ssh or automatic delivery of logs in the alert message. \\
\\
Another trend is that alerting manager implement a massive amount of different services for messaging that are used by modern developer teams like slack or telegram. Some of the bigger tools were also capable of REST or SOAP requests so that nearly every interface can be included. To keep overview of cascading failure ,some of the tools can group errors to one alert.
\\
In Visualization a big trend is the presentation of similar data in one single graph. These method provides a better overview of the system and leads to generalization of the data. Moreover the most tools use coloring to highlight warnings or errors. The big player on the market also implement a custom ordering engine, to drag around the graphs for an own sorting.
\\
We also saw some tendency in the database evolution. Databases are no longer just for storing single data points. They now have a greater added value by implementing there own querying language. Using this to purify the collected the data for visualization tools. The most established data format is JSON because the files can be fragmented over multiple nodes.

The field of collectors drifts towards multi functional interfaces. These trend is settled, because many system have to be integrated in already existing monitoring environments. In addition there are two types of collectors. Once the log collectors and secondly the metric one's. Both are necessary to monitor a complete system as a whole.

When we installed and tested the tools, all of the above trends were reflected in them. That is how we identified and worked out the trends. 


%With a more deeper look in the logging monitoring tools some more trends of the different tool types could be found and identified. 
\section{Conclusion}
At the beginning of this paper we asked the question which monitoring tools are the best. Clear is that there is no clear answer to this question. It always depends on the environment that is used, often some monitoring structure is already given and a new solutions have to be build around them.\\
To answer the question anyway we decided to recommend an combination from Zabbix and Grafana for Kubernetes monitoring. We think it is good, because the installation is very easy and can be done with one single command line. There are also different versions of the Zabbix portation. Some of them are more lightweighted or have more tools than the others. In terms of functions Zabbix nearly provides everything. To complete this functionality with usability and interoperability we recommend to configure a Grafana instance onto the REST api of Zabbix. This is easy done because Grafana has it own Zabbix plug-in. This plug-in provides a very nice dashboard view of the cluster metrics.
%TO DO
\\
If there is any reason to not go for an Zabbix installation or if a deeper look into the single nodes and pods is required, we could recommend a combination of Heapster with Influxdb and Grafana. The installation is a little bit more complex because all the tools have to be configured to work together. The huge benefit is that all tools are adopted by the Kubernetes Developers to work great with the cluster. This not only allows a very detailed look on every Pod that is deployed. The tools also offer a great performance even in small clusters. In special the Ram usage was lower than of any other tested tools.

\section{Future Work}
As a follow up to our work a deeper look into the field of log monitoring tools could provide a more efficient way of failure treatment. An important task to look is where micro service environments like Kubernetes or technology like docker store there logs. To get the best benefit out of the logs it is a major task to find a good storing engine for the quickly changing services. We think a clear mapping of the logs to the service informations is a main quest of the work.\\
As we only looked at solutions that are open-source there is the possibility to expand our work to the hole software market to take more tools into account. This work could discuss the question of benefits that open source has over paid tools for the new DevOps style of developing.


%
%
%\renewcommand{\appendixtocname}{Anhang}
%\renewcommand{\appendixname}{Anhang}
%\renewcommand{\appendixpagename}{Anhang}
\appendix
%\input{content/latex-tipps}

%\printindex

\printbibliography

\ifdeutsch
Alle URLs wurden zuletzt am 26.\,01.\,2018 geprüft.
\else
All links were last followed on January 26, 2018.
\fi

\pagestyle{empty}
\renewcommand*{\chapterpagestyle}{empty}
\Versicherung
\end{document}
