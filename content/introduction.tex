% !TeX spellcheck = en_US

\chapter{Introduction}
Nowadays its very Common in IT to have Distributed Systems in Location all over the Globe. To be able to provide the best user Experience its important to Monitor these Networks by only few People sitting in one or more Location. Important here is the Availability and Reliability also as the Response Time of the System. 

To tackle these kinds of Tasks Application Performance Management Tool were build. They are available in a wide range of Costs and Qualities. They differ a lot in their Architecture and style of tackling Problems. This is why we decided to make a Comparison of some large Open-Source tool stacks available on the Market 


\section*{Thesis Structure}
In the first part the Paper describes the general aspects of Monitoring and the general Metrics. The part also discusses the characteristics of the environments and their special interfaces. After the introduction the tools will be introduced on there own. As a conclusion to this work a General overview in form of some table is given. And a conclusion to the question of the best monitoring tool is given. 
\begin{description}
\item[Technical Data:] In this chapter all technical aspects of the Test environments and Tools are discussed. More over it provides our separation of the different types of tools 
\item[Tools:] All tested Tool stacks are listed and group by companies that developed and supports them. On the end of this Chapter is a short list of tool we tested and were not able to deploy on a cluster.
\item[Evaluation:]The tested tools compared in tables after there responsibility 
\item[Conclusion:] Our experience with the tools and a Answer to the question which stack is the best.
\section*{Goals}
Goal of the Study is to print out the benefits and disadvantages of the popular Open-Source Tools and Stacks  for monitoring available on the Market. The work wants to Illustrate the features and technologies of the tools to make it easier for the Reader to get an overview over the different Software approaches. In particular the tools will be tested in their ability to interact with modern Cloud technologies like Docker and Kubernetes. Furthermore they will be compared by their Ability to integrate in existing environments and support of common tools and Interfaces. Moreover the Cross Compatibility of the stacks will be tested to get the best out of the tool pool.
\end{description}
