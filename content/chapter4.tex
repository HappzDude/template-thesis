% !TeX spellcheck = en_US
\chapter{Evaluation} %Das ist nur ein Arbeitstitel
% \useunder{\uline}{\ul}{}
This chapter deals with the comparison of the tools by their responsibility. To explain some concepts, which we have seen a lot our research, we build an extra chapter (\cref{concepts}).
Each table shows the details the developer gives. They are ordered by the responsibility of tools. The information show in the tables were from the website of the tool providers.
In all tables below we printed out with a $ \checkmark $ and a $ X $ if we can agree with the information given by the developers.
\section{Comparison of concepts}
\label{concepts}
In the scope of this study we identified concepts to compare the tools. Most of them are self-explaining. The ones that are not self-explaining we listed below with a brief introduction.
\subsection{Push/Pull}
\label{push/pull}
There are two basic concepts of the exchange of informations between the collector and the database. Here they named push and pull. Push \cite{5557986} is a concept, were the collector  automatically starts sending data to the database as soon the system is started. The data is send continuously when the status changes. Important is that the collector knows or discovers the data destination.
Pull \cite{5557986} on the other hand means that the database sends requests to the active nodes. These requests are executed in certain intervals. These intervals can be modified and adjusted according to the system requirements. In this concept the database needs a discoverer to find all existing nodes
\subsection{REST}
\label{rest}
Representational State Transfer is a programming paradigm for distributed systems. It describes a way, how services in special eb services interact with each other. Important concepts of REST are oose coupling, interoperability and scalability. Any data in REST is made persistent in resources. These resources are addressable over an Uniform Resource Identifier. \\
To make an interface RESTful, it has to provide all HTTP methods like GET, PUT, POST, OPTIONS within a well known data format as XML or JSON.


\section{Collector}
The \cref{tab:Collector} compares the different collector tools tested in this paper. The tools are compared by RESTful (\cref{rest}), Multiple(Multi.) Input Formats, Plug-ins for own data and Push/Pull (\cref{push/pull}) Primary how the logs and metrics are transferred to the databases.
\begin{table}[H]
\centering
\begin{tabular}{lcccc}
\hline

Tool & RESTful & Multi. Input Formats      & Plug-ins for own data        & Push / Pull \\

\hline
Telegraf    & Yes ($ \checkmark $) & Yes    & Yes    &Pull    ($ \checkmark $)  \\
Beats  & Yes ($ \checkmark $)  & Yes & Yes  & Push ($ \checkmark $) \\
Logstash & Yes ($ \checkmark $)  & No & Yes & Push/Pull($\checkmark$)                         \\
Icinga2  & No  & No  & Yes  & Push \\
Prometheus  & No  & No  & Yes  & Pull ($ \checkmark $)/ Push\\
Zabbix & No  & No  & No  & Both \\
\hline                        
\end{tabular}
\caption{Collector-table - All tested collector tools are listed and compared}
\label{tab:Collector}
\end{table}

\section{Database}
\cref{tab:Database} shows the databases. First we checked if the tools have an authentication to the content. Furthermore we have selected the tools according to times-series-database. In addition to that, if there are several datasets at one time stamp. \\
\begin{table}[H]
\centering
\begin{tabular}{lccc}
	\hline
Tool & Database-Authentication     & Time-Series-Database          & Multi. Data Points        \\
\hline
InfluxDB  & Yes ($ \checkmark $) & Yes ($ \checkmark $)  & Yes ($ \checkmark $)\\
Elasticsearch & https ($ \checkmark $), credentials ($ X $) & Yes ($ \checkmark $) & Yes ($ \checkmark $)\\
Icinga2 & Yes ($ \checkmark $) & Yes ($ \checkmark $) & Yes ($ \checkmark $) \\
Prometheus& No & Yes ($ \checkmark $) & Yes ($ \checkmark $)\\
Zabbix& Yes ($ \checkmark $) & Yes ($ \checkmark $) & Yes ($ \checkmark $)\\
\hline
\end{tabular}
\caption{Database-table - All tested database tools are listed and compared}
\label{tab:Database}
\end{table}

\section{Visualization}
\cref{tab:Visualization} compares the different visualization tools. One aspect is the security of the tool, this is compared by LDAP-Authentication (\cref{ldap}). In general the tools are tested on the base of the graphs that they can display. We tested if they are scalable and it is possible to show multiple graphs in one. 
\begin{table}[H] 
\centering
\begin{tabular}{lccc}
	\hline
Tool & LDAP-Authentication         & Scalable graphs             & Overlapping graphs          \\
\hline
Chronograf & No & Yes & Yes \\
Kibana & Yes ($ \checkmark $) & Yes ($ \checkmark $) & Yes ($ \checkmark $)\\
Grafana & Yes ($ \checkmark $) & Yes ($ \checkmark $) & Yes ($ \checkmark $)\\
Zabbix & Yes ($ \checkmark $) & Yes ($ \checkmark $)  & Yes  ($ \checkmark $)\\
\hline
\end{tabular}
\caption{Visualization-table - All tested visualization tools are listed and compared}
\label{tab:Visualization}
\end{table}

\section{Alerting}
The last \cref{tab:Alerting} and \cref{tab:Alertingcont} compares the altering tools. These tools were tested if they can send HTTP requests. Moreover, when many identical alerts arrive, whether they are groupable. Furthermore we tested the most common messaging tools for sending alerts and if the tools offer an interface to integrate any third-party-tool for messaging. Finally, the tools have been tested for triggering events like scripts when an alert is received.
\begin{table}[H]
	\centering
	\begin{tabular}{lcccc}
		\hline
Tool & HTTP requests & Grouping of alerts & E-Mail & Messangers \\
\hline
Kapacitor & Yes  & No & Yes & Yes \\
Elastalert & No & No & Yes ($ \checkmark $) & Yes ($ \checkmark $) \\
Grafana & Yes ($ \checkmark $) & Yes ($ \checkmark $) & Yes ($ \checkmark $) & Yes ($ \checkmark $) \\
Zabbix & Yes ($ \checkmark $) & Yes ($ \checkmark $) & Yes ($ \checkmark $) & Yes ($ \checkmark $) \\
Prometheus & Yes ($ \checkmark $) & Yes ($ X $) & Yes ($ X $) & Yes ($ X $)\\
		\hline
	\end{tabular}
	\caption{Alerting-table - All tested alerting tools are listed and compared}
	\label{tab:Alerting}
\end{table}

% Please add the following required packages to your document preamble:
% \usepackage[table,xcdraw]{xcolor}
% If you use beamer only pass "xcolor=table" option, i.e. \documentclass[xcolor=table]{beamer}
\begin{table}[H]
	\centering
	\begin{tabular}{lcc}
		\hline
		Tool & Custom alerts & Action on alert \\
		\hline
		Kapacitor                    & Yes ($ \checkmark $)       & No\\
		Elastalert                   & Yes ($ \checkmark $) & No\\
		Grafana                      & Yes ($ \checkmark $)          & Yes ($ \checkmark $)\\
		Zabbix  & Yes ($ \checkmark $) & Yes ($ \checkmark $)\\
		Prometheus & Yes ($ \checkmark $) & No \\
		\hline 
	\end{tabular}
	\caption{Alerting-table - \cref{tab:Alerting} cont.}
	\label{tab:Alertingcont}
\end{table}

