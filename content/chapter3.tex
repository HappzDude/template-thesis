% !TeX spellcheck = en_US
 
\chapter{Tools} %Das ist nur ein Abreitstitel
\section{Failed Tools}%Ebenfalls nur ein arbeitstitel
In the Process of Developing and Evaluating the APM we Discoverd a bunch of Tools that we were not able to install even that they clamed to be optimized to work on Kubernetes .
In this Paragraph all the tools we wanted to include in our Report but doesn't are mentioned with a quick description of the Failure.
\subsection{Graphite}
Graphite is mainly for storing and Graphing data and metrics, but brings also tools that are able to collect these Metrics from the system. By the Developer it self there is no Kubernetes installation Provide but there are diverse approaches by third party members to make it runnable on a Cluster. We have tested the Repository from nanit\\(\url{ https://github.com/nanit/kubernetes-graphite-cluster},11.12.2017) to get Graphite running with StatsD (\url{https://github.com/etsy/statsd.git},11.12.2017) as a metric collection tool. The Repo doesn't provide a yaml file by it self to install all the tools. The instruction leads the user to export some Variables needed for the installation. After that a deploy command is provided that pulls the docker repo and than installs it with kubectl on the Cluster. As we tried to execute this command a fail was thrown, The Node Replicas were empty, so no further commands are executable. As we were not able to install the tool on multiple Kubernetes Clusters, we installed the tool as on the Webside advertised on a Ubuntu System directly. With this installation of Graphit a time-series Data,an Monitoring and an Alerting tool is included. On the test System the Monitoring System gave values that differs from the Linux intern monitoring in values like Cpu usage or RAM. As a Solution to all this Difficulties we decided to not perform further test on the tool.

\subsection{Icinga/searchlight}
Because there was no direkt support of Icinga for Kubernetes by the Developer. We found a tool that is Build on a Icinga basic and claims to work fine on a Kubernetes Cluster. Its Developed by appscode and named searchlight. As we tried to  