% !TeX spellcheck = en_US

\chapter{Conclusion}\label{chap:conclusion}
\section{General Trends}
Most of the systems we evaluated or considered evaluating were metric monitoring stacks. In general most tools detect not the cause of the failure. Instead the effect is recognized and can be treated many manualy. With some tools we saw a trend towards more automation. These tools had functions like automated script triggering over ssh or automatic delivery of logs in the alert message. \\
\\
Another trend is that alerting manager implement a massive amount of different services for messaging that are used by modern developer teams like slack or telegram. Some of the bigger tools were also capable of REST or SOAP requests so that nearly every interface can be included. To keep overview of cascading failure ,some of the tools can group errors to one alert.
\\
In Visualization a big trend is the presentation of similar data in one single graph. These method provides a better overview of the system and leads to generalization of the data. Moreover the most tools use coloring to highlight warnings or errors. The big player on the market also implement a custom ordering engine, to drag around the graphs for an own sorting.
\\
We also saw some tendency in the database evolution. Databases are no longer just for storing single data points. They now have a greater added value by implementing there own querying language. Using this to purify the collected the data for visualization tools. The most established data format is JSON because the files can be fragmented over multiple nodes.

The field of collectors drifts towards multi functional interfaces. These trend is settled, because many system have to be integrated in already existing monitoring environments. 
\section{Conclusion}
At the beginning of the paper we asked the question which monitoring tool/tools is the best. Clear is that there is no single answer to this question. It always depends on the environment that is used. Often some monitoring structure is already given and new solutions have to be build around them.\\
To answer the Question anyway we decided to recommend an combination from Zabbix and Grafana for Kubernetes Monitoring. We think its good, because the installation is very easy and can be done with one single command line. There are also different versions of the Zabbix portation. Some of them are more lightwighted or have more tools than overs. In terms of functions Zabbix nearly provides everything. To complete this functionality with usability and interoperability we recommend to configure an Grafana instance onto the REST Api of Zabbix. This can easyl be done because Grafana has its own Zabbix plugin. This plugin than provides a very nice Dashboard view of the cluster Metrics.
%TO DO
\\
If there is any reason to not go for an Zabbix installation. Or if a deeper look into the single nodes an pods is required, we could recommend a combination of Heapster with Influxdb and Grafana. The installation is a little bit more complex because all the tools have to be configured to work together. The huge benefit is that all tools are adopted by the Kubernetes Developers to work great with the cluster. This not only allows a very detailed look on every Pod that is deployed. The tools also offer a great performance even in small clusters. In special the Ram usage was lower than of any other tested tools.

\section*{Future Work}
As a follow up to our work a deeper look into the filed of log monitoring tools could provide a more efficient way of failure treatment.An Important thing to look is where micro service environments like Kubernetes or technology like docker sore there logs. To get the best benefit out of the logs it is a major task to find a good storing engine for the quickly changing services. We think a clear mapping of the logs to the service informations is a main quest of the work we suggest.\\
As we only looked at solutions that are open-source there is the possible to expand our work to the hole software market to take more tools into account. This work could discuss the question of benefits that open source has over paid tools for the new DevOps style of developing.

